\documentclass[b5paper]{book}
\usepackage{hyperref}
\usepackage{makeidx}
\usepackage{amssymb}
\usepackage{color}
\usepackage{alltt}
\usepackage{graphicx}
\usepackage{layout}
\def\union{\cup}
\def\intersect{\cap}
\def\getsrandom{\stackrel{\rm R}{\gets}}
\def\cross{\times}
\def\cat{\hspace{0.5em} \| \hspace{0.5em}}
\def\catn{$\|$}
\def\divides{\hspace{0.3em} | \hspace{0.3em}}
\def\nequiv{\not\equiv}
\def\approx{\raisebox{0.2ex}{\mbox{\small $\sim$}}}
\def\lcm{{\rm lcm}}
\def\gcd{{\rm gcd}}
\def\log{{\rm log}}
\def\ord{{\rm ord}}
\def\abs{{\mathit abs}}
\def\rep{{\mathit rep}}
\def\mod{{\mathit\ mod\ }}
\renewcommand{\pmod}[1]{\ ({\rm mod\ }{#1})}
\newcommand{\floor}[1]{\left\lfloor{#1}\right\rfloor}
\newcommand{\ceil}[1]{\left\lceil{#1}\right\rceil}
\def\Or{{\rm\ or\ }}
\def\And{{\rm\ and\ }}
\def\iff{\hspace{1em}\Longleftrightarrow\hspace{1em}}
\def\implies{\Rightarrow}
\def\undefined{{\rm ``undefined"}}
\def\Proof{\vspace{1ex}\noindent {\bf Proof:}\hspace{1em}}
\let\oldphi\phi
\def\phi{\varphi}
\def\Pr{{\rm Pr}}
\newcommand{\str}[1]{{\mathbf{#1}}}
\def\F{{\mathbb F}}
\def\N{{\mathbb N}}
\def\Z{{\mathbb Z}}
\def\R{{\mathbb R}}
\def\C{{\mathbb C}}
\def\Q{{\mathbb Q}}
\definecolor{DGray}{gray}{0.5}
\newcommand{\emailaddr}[1]{\mbox{$<${#1}$>$}}
\def\twiddle{\raisebox{0.3ex}{\mbox{\tiny $\sim$}}}
\def\gap{\vspace{0.5ex}}
\makeindex
\begin{document}
\frontmatter
\pagestyle{empty}
\title{LibTomPoly User Manual \\ v0.02}
\author{Tom St Denis \\ tomstdenis@iahu.ca}
\maketitle
This text and library are  hereby placed in the public domain.  This book has been 
formatted for B5 [176x250] paper using the \LaTeX{} {\em book} macro package.

\vspace{10cm}

\begin{flushright}Open Source.  Open Academia.  Open Minds.

\mbox{ }

Tom St Denis,

Ontario, Canada
\end{flushright}

\tableofcontents
\listoffigures
\mainmatter
\pagestyle{headings}
\chapter{Introduction}
\section{What is LibTomPoly?}
LibTomPoly is a public domain open source library to provide polynomial basis arithmetic.  It uses the public domain
library LibTomMath (not included) for the integer arithmetic and extends the functonality to provide polynomial arithmetic.

Technically speaking the library allows the user to perform arithmetic on elements from the group $GF(p)[x]$ and to 
a lesser extent (this will change in the future) over $\Z[x]$.  Essentially the math you can do with integers (including
forming rings and fields) you can do with with polynomials and now you can do with LibTomPoly.

\section{License}
LibTomPoly is public domain.  Enjoy.

\section{Terminology}
Throughout this manual and within the library there will be some terminology that not everyone is familiar with.  It is afterall
weird math.

\begin{figure}[here]
\begin{center}
\begin{small}
\begin{tabular}{|l|l|}
\hline \textbf{Term} & \textbf{Definition} \\
\hline monic polynomial & A polynomial where the leading coefficient is a one. \\
\hline irreducible polynomial & A polynomial that has no factors in a given group.  \\
                              & For instance, $x^2 + 4$ is irreducible in $\Z[x]$ but not \\
                              & in $GF(17)[x]$ since $x^2 + 4 = (x+8)(x+9) \mbox{ (mod }17\mbox{)}$.  \\
\hline primitive polynomial   & An irreducible polynomial which generates all of \\
                              & elements of a given field (e.g. $GF(p)[x]/v(x)$) \\
\hline characteristic         & An integer $k$ such that $k \cdot p(x) \equiv 0$  \\
\hline deg()                  & Functon returns degree of polynomial, e.g. deg($x^6 + x^3 + 1$) = 6 \\
\hline
\end{tabular}
\end{small}
\end{center}
\caption{Terminology}
\end{figure} 

\section{Building the Library}
The library is not ready for production yet but you can test out the library manually if you want.  To build the library 
simply type

\begin{alltt}
make
\end{alltt}

Which will build ``libtompoly.a''.  As of yet there is no MSVC makefile but one will be released shortly.  To build against
this library include ``tompoly.h'' and link against ``libtompoly.a''.

To build the included demo type 

\begin{alltt}
make demo
\end{alltt}

Which will build ``demo'' in the current directory.  The demo is not interactive and produces results which must be manually
inspected.

\chapter{Getting Started}

\section{The LibTomMath Connection}
LibTomPoly is really just an extension of LibTomMath\footnote{\url{http://math.libtomcrypt.org}}.  As such the library has 
been designed in much the same way as far as argument passing and error handling events are concerned.  The reader is 
encouraged to become familiar with LibTomMath before diving into LibTomPoly.

\section{The pb\_poly structure}
A polynomial can characterized by a few variables.  Given the C structure as follows

\begin{alltt}
typedef struct \{
   int    used,                  /* number of terms */
          alloc;                 /* number of terms available (total) */
   mp_int characteristic,        /* characteristic, zero if not finite */
          *terms;                /* terms of polynomial */
\} pb_poly;
\end{alltt}

\begin{enumerate}
   \item The \textbf{used} member indicates how many terms of the \textbf{terms} array are used to represent the polynomial.
   \item The \textbf{alloc} member indicates the size of the \textbf{terms} array.  Also note that even if \textbf{used}
         is less than \textbf{alloc} the mp\_ints above \textbf{used} in the array must be set to a valid representation 
         of zero.
   \item The \textbf{characteristic} member is an mp\_int representing the characteristic of the polynomial.  If the desire is to
         have a null characteristic (e.g. $\Z[x]$) this element must still be initialized to a valid representation
         of zero. 
   \item The \textbf{terms} member is a dynamically sized array of mp\_int values which represent the coefficients for
         the terms of the polynomial.  They start from least to most significant degree.  E.g. $p(x) = \sum_{i=0}^{used-1} terms_i \cdot x^i$.
\end{enumerate}

\section{Return Codes}
The library uses the return codes from LibTomMath.  They are

\index{MP\_OKAY}\index{MP\_YES}\index{MP\_NO}\index{MP\_VAL}\index{MP\_MEM}
\begin{figure}[here!]
\begin{center}
\begin{small}
\begin{tabular}{|l|l|}
\hline \textbf{Code} & \textbf{Meaning} \\
\hline MP\_OKAY & The function succeeded. \\
\hline MP\_VAL  & The function input was invalid. \\
\hline MP\_MEM  & Heap memory exhausted. \\
\hline &\\
\hline MP\_YES  & Response is yes. \\
\hline MP\_NO   & Response is no. \\
\hline
\end{tabular}
\end{small}
\end{center}
\caption{Return Codes}
\end{figure}

\section{Function Argument Passing}
Just like LibTomMath the arguments are meant to be read left to right where the destination is on the right.  Consider
the following.

\begin{alltt}
pb_add(a, b, c);   /* c = a + b */
pb_mul(a, b, c);   /* c = a * b */
\end{alltt}

Also like LibTomMath input arguments can be specified as output arguments.  Consider.

\begin{alltt}
pb_mul(a, b, a);   /* a = a * b */
pb_gcd(a, b, b);   /* b = (a, c) */
\end{alltt}

However, polynomial math raises another consideration.  The characteristic of the result is taken from the right most
argument passed to the function.  Not all functions will return an error code if the characteristics of the inputs
do not match so it's important to keep this in mind.  In general the results are undefined if not all of the polynomials
have identical characteristics.

\section{Initializing Polynomials}
In order to use a pb\_poly structure with one of the functions in this library the structure must be initialized.  
There are three functions provided to initialize pb\_poly structures.

\subsection{Default Initialization}
\index{pb\_init}
\begin{alltt}
int pb_init(pb_poly *a, mp_int *characteristic);
\end{alltt}
This will initialize ``a'' with the given ``characteristic'' such that the polynomial represented is a constant zero.  
The mp\_int characteristic must be a valid initialized mp\_int even if a characteristic of zero is desired.  By default, 
the polynomial will be initialized so there are ``PB\_TERMS'' terms available.  This will grow automatically as required 
by the other functions.

\subsection{Initilization of Given Size}
\index{pb\_init\_size}
\begin{alltt}
int pb_init_size(pb_poly *a, mp_int *characteristic, int size);
\end{alltt}
This behaves similar to pb\_init() except it will allocate ``size'' terms to initialize instead of ``PB\_TERMS''.  This
is useful if you happen to know in advance how many terms you want.

\subsection{Initilization of a Copy}
\index{pb\_init\_copy}
\begin{alltt}
int pb_init_copy(pb_poly *a, pb_poly *b);
\end{alltt}

This will initialize ``a'' so it is a verbatim copy of ``b''.  It will copy the characteristic and all of the terms
from ``b'' into ``a''.

\subsection{Freeing a Polynomial}
\index{pb\_clear}
\begin{alltt}
int pb_clear(pb_poly *a);
\end{alltt}

This will free all the memory required by ``a'' and mark it as been freed.  You can repeatedly pb\_clear() the same
pb\_poly safely.

\chapter{Basic Operations}
\section{Comparison}
Comparisions with polynomials is a bit less intuitive then with integers.  Is $x^2 + 3$ greater than $x^2 + x + 4$?  To
create a rational form of comparison the following comparison codes were designed.

\begin{figure}[here]
\begin{small}
\begin{center}
\begin{tabular}{|l|l|}
\hline \textbf{Code} & \textbf{Meaning} \\
\hline PB\_EQ & The polynomials are exactly equal \\
\hline PB\_DEG\_LT & The left polynomial has a lower degree than the right. \\
\hline PB\_DEG\_EQ & Both have the same degree. \\
\hline PB\_DEG\_GT & The left polynomial has a higher degree than the right. \\
\hline
\end{tabular}
\end{center}
\end{small}
\caption{Compare Codes}
\end{figure}

\index{pb\_cmp}
\begin{alltt}
int pb_cmp(pb_poly *a, pb_poly *b);
\end{alltt}

This will compare the polynomial ``a'' to the left of the polynomial ``b''.  It will return one of the four
codes listed above.  Note that the function does not compare the characteristics.  So if $a \in GF(17)[x]$ and 
$b \in GF(11)[x]$ were both equal to $x^2 + 3$ they would compare to PB\_EQ.  Whereas $x^3 + 4$
would compare to PB\_DEG\_LT, $x^1 + 7$ would compare to $PB\_DEG\_GT$ and $x^2 + 7$ would compare to $PB\_DEG\_EQ$\footnote{If the polynomial $a$ were on the left for all three cases.}.

\section{Copying and Swapping}
\index{pb\_copy}
\begin{alltt}
int pb_copy(pb_poly *src, pb_poly *dest);
\end{alltt}
This will copy the polynomial from ``src'' to ``dest'' verbatim.

\index{pb\_exch}
\begin{alltt}
int pb_exch(pb_poly *a, pb_poly *b);
\end{alltt}
This will exchange the contents of ``a'' with ``b''.  

\section{Multiplying and Dividing by $x$}
\index{pb\_lshd} \index{pb\_rshd}
\begin{alltt}
int pb_lshd(pb_poly *a, int i);
int pb_rshd(pb_poly *a, int i);
\end{alltt}
These will multiply (or divide, respectfully) the polynomial ``a'' by $x^i$.  If $i \le 0$ the functions return without
performing any operation.  For example,

\begin{alltt}
pb_lshd(a, 2);  /* a(x) = a(x) * x^2 */
pb_rshd(a, 7);  /* a(x) = a(x) / x^7 */
\end{alltt}

\chapter{Basic Arithmetic}
\section{Addition, Subtraction and Multiplication}
\index{pb\_add} \index{pb\_sub}
\begin{alltt}
int pb_add(pb_poly *a, pb_poly *b, pb_poly *c);
int pb_sub(pb_poly *a, pb_poly *b, pb_poly *c);
int pb_mul(pb_poly *a, pb_poly *b, pb_poly *c);
\end{alltt}

These will add (subtract or multiply, respectfully) the polynomial ``a'' and polynomial ``b'' and store the result in 
polynomial ``c''.  The characteristic from  ``c'' is used to calculate the result.  Note that the coefficients of ``c'' 
will always be positive provided the characteristic of ``c'' is greater than zero.  

Quick examples of usage.

\begin{alltt}
pb_add(a, b, c);  /* c = a + b */
pb_sub(b, a, c);  /* c = b - a */
pb_mul(c, a, a);  /* a = c * a */
\end{alltt}

\section{Division}
\index{pb\_div}
\begin{alltt}
int pb_div(pb_poly *a, pb_poly *b, pb_poly *c, pb_poly *d);
\end{alltt}
This will divide the polynomial ``a'' by ``b'' and store the quotient in ``c'' and remainder in ``d''.  That is

\begin{equation}
b(x) \cdot c(x) + d(x) = a(x)
\end{equation}

The value of $deg(d(x))$ is always less than $deg(b(x))$.  Either of ``c'' and ``d'' can be set to \textbf{NULL} to 
signify their value is not desired.  This is useful if you only want the quotient or remainder but not both.  

Since one of the destinations can be \textbf{NULL} the characteristic of the result is taken from ``b''.  The function
will return an error if the characteristic of ``a'' differs from that of ``b''.  

This function is defined for polynomials in $GF(p)[x]$ only.  A routine pb\_zdiv()\footnote{To be written!} allows the 
division of polynomials in $\Z[x]$.  

\section{Modular Functions}
\index{pb\_addmod} \index{pb\_submod} \index{pb\_mulmod}
\begin{alltt}
int pb_addmod(pb_poly *a, pb_poly *b, pb_poly *c, pb_poly *d);
int pb_submod(pb_poly *a, pb_poly *b, pb_poly *c, pb_poly *d);
int pb_mulmod(pb_poly *a, pb_poly *b, pb_poly *c, pb_poly *d);
\end{alltt}

These add (subtract or multiply respectfully) the polynomial ``a'' and the polynomial ``b'' modulo the polynomial ``c''
and store the result in the polynomial ``d''.  

\chapter{Algebraic Functions}

\section{Monic Reductions}
\index{pb\_monic}
\begin{alltt}
int pb_monic(pb_poly *a, pb_poly *b)
\end{alltt}
Makes ``b'' the monic representation of ``a'' by ensuring the most significant coefficient is one.  Only defined
over $GF(p)[x]$.  Note that this is not a straight copy to ``b'' so you must ensure the characteristic of the two
are equal before you call the function\footnote{Note that $a == b$ is acceptable as well.}.  Monic polynomials
are related to their original polynomial through an integer $k$ as follows

\begin{equation}
a(x) \cdot k^{-1} \equiv b(x)
\end{equation}

\section{Extended Euclidean Algorithm}
\index{pb\_exteuclid}
\begin{alltt}
int pb_exteuclid(pb_poly *a, pb_poly *b, 
                 pb_poly *U1, pb_poly *U2, pb_poly *U3);
\end{alltt}

This will compute the Euclidean algorithm and find values ``U1'', ``U2'', ``U3'' such that 

\begin{equation}
a(x) \cdot U1(x) + b(x) \cdot U2(x) = U3(x)
\end{equation}

The value of ``U3'' is reduced to a monic polynomial.  The three destination variables are all optional and can
be specified as \textbf{NULL} if they are not desired.

\section{Greatest Common Divisor}
\index{pb\_gcd}
\begin{alltt}
int pb_gcd(pb_poly *a, pb_poly *b, pb_poly *c);
\end{alltt}
This finds the monic greatest common divisor of the two polynomials ``a'' and ``b'' and store the result in ``c''.  The
operation is only defined over $GF(p)[x]$.  

\section{Modular Inverse}
\index{pb\_invmod}
\begin{alltt}
int pb_invmod(pb_poly *a, pb_poly *b, pb_poly *c);
\end{alltt}
This finds the modular inverse of ``a'' modulo ``b'' and stores the result in ``c''.  The operation is only defined over 
$GF(p)[x]$.  If the operation succeed then the following congruency should hold true.

\begin{equation}
a(x)c(x) \equiv 1 \mbox{ (mod }b(x)\mbox{)}
\end{equation}

\section{Modular Exponentiation}
\index{pb\_exptmod}
\begin{alltt}
int pb_exptmod (pb_poly * G, mp_int * X, pb_poly * P, pb_poly * Y);
\end{alltt}

This raise ``G'' to the power of ``X'' modulo ``P'' and stores the result in ``Y''. Or as a congruence

\begin{equation}
Y(x) \equiv G(x)^X \mbox{ (mod }P(x)\mbox{)}
\end{equation}

Where ``X'' can be negative\footnote{But in that case $G^{-1}(x)$ must exist modulo $P(x)$.} or positive.  This function
is only defined over $GF(p)[x]$.  

\section{Irreducibility Testing}
\index{pb\_isirreduc}
\begin{alltt}
int pb_isirreduc(pb_poly *a, int *res);
\end{alltt}
Sets ``res'' to MP\_YES is ``a'' is irreducible (only for $GF(p)[x]$) otherwise sets ``res'' to MP\_NO.  

\documentclass[b5paper]{book}
\usepackage{hyperref}
\usepackage{makeidx}
\usepackage{amssymb}
\usepackage{color}
\usepackage{alltt}
\usepackage{graphicx}
\usepackage{layout}
\def\union{\cup}
\def\intersect{\cap}
\def\getsrandom{\stackrel{\rm R}{\gets}}
\def\cross{\times}
\def\cat{\hspace{0.5em} \| \hspace{0.5em}}
\def\catn{$\|$}
\def\divides{\hspace{0.3em} | \hspace{0.3em}}
\def\nequiv{\not\equiv}
\def\approx{\raisebox{0.2ex}{\mbox{\small $\sim$}}}
\def\lcm{{\rm lcm}}
\def\gcd{{\rm gcd}}
\def\log{{\rm log}}
\def\ord{{\rm ord}}
\def\abs{{\mathit abs}}
\def\rep{{\mathit rep}}
\def\mod{{\mathit\ mod\ }}
\renewcommand{\pmod}[1]{\ ({\rm mod\ }{#1})}
\newcommand{\floor}[1]{\left\lfloor{#1}\right\rfloor}
\newcommand{\ceil}[1]{\left\lceil{#1}\right\rceil}
\def\Or{{\rm\ or\ }}
\def\And{{\rm\ and\ }}
\def\iff{\hspace{1em}\Longleftrightarrow\hspace{1em}}
\def\implies{\Rightarrow}
\def\undefined{{\rm ``undefined"}}
\def\Proof{\vspace{1ex}\noindent {\bf Proof:}\hspace{1em}}
\let\oldphi\phi
\def\phi{\varphi}
\def\Pr{{\rm Pr}}
\newcommand{\str}[1]{{\mathbf{#1}}}
\def\F{{\mathbb F}}
\def\N{{\mathbb N}}
\def\Z{{\mathbb Z}}
\def\R{{\mathbb R}}
\def\C{{\mathbb C}}
\def\Q{{\mathbb Q}}
\definecolor{DGray}{gray}{0.5}
\newcommand{\emailaddr}[1]{\mbox{$<${#1}$>$}}
\def\twiddle{\raisebox{0.3ex}{\mbox{\tiny $\sim$}}}
\def\gap{\vspace{0.5ex}}
\makeindex
\begin{document}
\frontmatter
\pagestyle{empty}
\title{LibTomPoly User Manual \\ v0.04}
\author{Tom St Denis \\ tomstdenis@iahu.ca}
\maketitle
This text and library are  hereby placed in the public domain.  This book has been 
formatted for B5 [176x250] paper using the \LaTeX{} {\em book} macro package.

\vspace{10cm}

\begin{flushright}Open Source.  Open Academia.  Open Minds.

\mbox{ }

Tom St Denis,

Ontario, Canada
\end{flushright}

\tableofcontents
\listoffigures
\mainmatter
\pagestyle{headings}
\chapter{Introduction}
\section{What is LibTomPoly?}
LibTomPoly is a public domain open source library to provide polynomial basis arithmetic.  It uses the public domain
library LibTomMath (not included) for the integer arithmetic and extends the functonality to provide polynomial arithmetic.

Technically speaking the library allows the user to perform arithmetic on elements from the group $GF(p)[x]$ and to 
a lesser extent (this will change in the future) over $\Z[x]$.  Essentially the math you can do with integers (including
forming rings and fields) you can do with with polynomials and now you can do with LibTomPoly.

\section{License}
LibTomPoly is public domain.  Enjoy.

\section{Terminology}
Throughout this manual and within the library there will be some terminology that not everyone is familiar with.  It is afterall
weird math.

\begin{figure}[here]
\begin{center}
\begin{small}
\begin{tabular}{|l|l|}
\hline \textbf{Term} & \textbf{Definition} \\
\hline monic polynomial & A polynomial where the leading coefficient is a one. \\
\hline irreducible polynomial & A polynomial that has no factors in a given group.  \\
                              & For instance, $x^2 + 4$ is irreducible in $\Z[x]$ but not \\
                              & in $GF(17)[x]$ since $x^2 + 4 = (x+8)(x+9) \mbox{ (mod }17\mbox{)}$.  \\
\hline primitive polynomial   & An irreducible polynomial which generates all of \\
                              & elements of a given field (e.g. $GF(p)[x]/v(x)$) \\
\hline characteristic         & An integer $k$ such that $k \cdot p(x) \equiv 0$  \\
\hline deg()                  & Functon returns degree of polynomial, e.g. deg($x^6 + x^3 + 1$) = 6 \\
\hline
\end{tabular}
\end{small}
\end{center}
\caption{Terminology}
\end{figure} 

\section{Building the Library}
The library is not ready for production yet but you can test out the library manually if you want.  To build the library 
simply type

\begin{alltt}
make
\end{alltt}

Which will build ``libtompoly.a''.  To build a Win32 library with MSVC type

\begin{alltt}
nmake -f makefile.msvc
\end{alltt}

To build against this library include ``tompoly.h'' and link against ``libtompoly.a'' (or tommath.lib as appropriate).
To build the included demo type 

\begin{alltt}
make demo
\end{alltt}

Which will build ``demo'' in the current directory.  The demo is not interactive and produces results which must be manually
inspected.

\chapter{Getting Started}

\section{The LibTomMath Connection}
LibTomPoly is really just an extension of LibTomMath\footnote{\url{http://math.libtomcrypt.org}}.  As such the library has 
been designed in much the same way as far as argument passing and error handling events are concerned.  The reader is 
encouraged to become familiar with LibTomMath before diving into LibTomPoly.

\section{The pb\_poly structure}
A polynomial can characterized by a few variables.  Given the C structure as follows

\begin{alltt}
typedef struct \{
   int    used,                  /* number of terms */
          alloc;                 /* number of terms available (total) */
   mp_int characteristic,        /* characteristic, zero if not finite */
          *terms;                /* terms of polynomial */
\} pb_poly;
\end{alltt}

\begin{enumerate}
   \item The \textbf{used} member indicates how many terms of the \textbf{terms} array are used to represent the polynomial.
   \item The \textbf{alloc} member indicates the size of the \textbf{terms} array.  Also note that even if \textbf{used}
         is less than \textbf{alloc} the mp\_ints above \textbf{used} in the array must be set to a valid representation 
         of zero.
   \item The \textbf{characteristic} member is an mp\_int representing the characteristic of the polynomial.  If the desire is to
         have a null characteristic (e.g. $\Z[x]$) this element must still be initialized to a valid representation
         of zero. 
   \item The \textbf{terms} member is a dynamically sized array of mp\_int values which represent the coefficients for
         the terms of the polynomial.  They start from least to most significant degree.  E.g. $p(x) = \sum_{i=0}^{used-1} terms_i \cdot x^i$.
\end{enumerate}

\section{Return Codes}
The library uses the return codes from LibTomMath.  They are

\index{MP\_OKAY}\index{MP\_YES}\index{MP\_NO}\index{MP\_VAL}\index{MP\_MEM}
\begin{figure}[here!]
\begin{center}
\begin{small}
\begin{tabular}{|l|l|}
\hline \textbf{Code} & \textbf{Meaning} \\
\hline MP\_OKAY & The function succeeded. \\
\hline MP\_VAL  & The function input was invalid. \\
\hline MP\_MEM  & Heap memory exhausted. \\
\hline &\\
\hline MP\_YES  & Response is yes. \\
\hline MP\_NO   & Response is no. \\
\hline
\end{tabular}
\end{small}
\end{center}
\caption{Return Codes}
\end{figure}

\section{Function Argument Passing}
Just like LibTomMath the arguments are meant to be read left to right where the destination is on the right.  Consider
the following.

\begin{alltt}
pb_add(a, b, c);   /* c = a + b */
pb_mul(a, b, c);   /* c = a * b */
\end{alltt}

Also like LibTomMath input arguments can be specified as output arguments.  Consider.

\begin{alltt}
pb_mul(a, b, a);   /* a = a * b */
pb_gcd(a, b, b);   /* b = (a, b) */
\end{alltt}

However, polynomial math raises another consideration.  The characteristic of the result is taken from the right most
argument passed to the function.  Not all functions will return an error code if the characteristics of the inputs
do not match so it's important to keep this in mind.  In general the results are undefined if not all of the polynomials
have identical characteristics.

\section{Initializing Polynomials}
In order to use a pb\_poly structure with one of the functions in this library the structure must be initialized.  
There are three functions provided to initialize pb\_poly structures.

\subsection{Default Initialization}
\index{pb\_init}
\begin{alltt}
int pb_init(pb_poly *a, mp_int *characteristic);
\end{alltt}
This will initialize ``a'' with the given ``characteristic'' such that the polynomial represented is a constant zero.  
The mp\_int characteristic must be a valid initialized mp\_int even if a characteristic of zero is desired.  By default, 
the polynomial will be initialized so there are ``PB\_TERMS'' terms available.  This will grow automatically as required 
by the other functions.

\subsection{Initilization of Given Size}
\index{pb\_init\_size}
\begin{alltt}
int pb_init_size(pb_poly *a, mp_int *characteristic, int size);
\end{alltt}
This behaves similar to pb\_init() except it will allocate ``size'' terms to initialize instead of ``PB\_TERMS''.  This
is useful if you happen to know in advance how many terms you want.

\subsection{Initilization of a Copy}
\index{pb\_init\_copy}
\begin{alltt}
int pb_init_copy(pb_poly *a, pb_poly *b);
\end{alltt}

This will initialize ``a'' so it is a verbatim copy of ``b''.  It will copy the characteristic and all of the terms
from ``b'' into ``a''.

\subsection{Freeing a Polynomial}
\index{pb\_clear}
\begin{alltt}
int pb_clear(pb_poly *a);
\end{alltt}

This will free all the memory required by ``a'' and mark it as been freed.  You can repeatedly pb\_clear() the same
pb\_poly safely.

\chapter{Basic Operations}
\section{Comparison}
Comparisions with polynomials is a bit less intuitive then with integers.  Is $x^2 + 3$ greater than $x^2 + x + 4$?  To
create a rational form of comparison the following comparison codes were designed.

\begin{figure}[here]
\begin{small}
\begin{center}
\begin{tabular}{|l|l|}
\hline \textbf{Code} & \textbf{Meaning} \\
\hline PB\_EQ & The polynomials are exactly equal \\
\hline PB\_DEG\_LT & The left polynomial has a lower degree than the right. \\
\hline PB\_DEG\_EQ & Both have the same degree. \\
\hline PB\_DEG\_GT & The left polynomial has a higher degree than the right. \\
\hline
\end{tabular}
\end{center}
\end{small}
\caption{Compare Codes}
\end{figure}

\index{pb\_cmp}
\begin{alltt}
int pb_cmp(pb_poly *a, pb_poly *b);
\end{alltt}

This will compare the polynomial ``a'' to the left of the polynomial ``b''.  It will return one of the four
codes listed above.  Note that the function does not compare the characteristics.  So if $a \in GF(17)[x]$ and 
$b \in GF(11)[x]$ were both equal to $x^2 + 3$ they would compare to PB\_EQ.  Whereas $x^3 + 4$
would compare to PB\_DEG\_LT, $x^1 + 7$ would compare to $PB\_DEG\_GT$ and $x^2 + 7$ would compare to $PB\_DEG\_EQ$\footnote{If the polynomial $a$ were on the left for all three cases.}.

\section{Copying and Swapping}
\index{pb\_copy}
\begin{alltt}
int pb_copy(pb_poly *src, pb_poly *dest);
\end{alltt}
This will copy the polynomial from ``src'' to ``dest'' verbatim.

\index{pb\_exch}
\begin{alltt}
int pb_exch(pb_poly *a, pb_poly *b);
\end{alltt}
This will exchange the contents of ``a'' with ``b''.  

\section{Multiplying and Dividing by $x$}
\index{pb\_lshd} \index{pb\_rshd}
\begin{alltt}
int pb_lshd(pb_poly *a, int i);
int pb_rshd(pb_poly *a, int i);
\end{alltt}
These will multiply (or divide, respectfully) the polynomial ``a'' by $x^i$.  If $i \le 0$ the functions return without
performing any operation.  For example,

\begin{alltt}
pb_lshd(a, 2);  /* a(x) = a(x) * x^2 */
pb_rshd(a, 7);  /* a(x) = a(x) / x^7 */
\end{alltt}

\chapter{Basic Arithmetic}
\section{Addition, Subtraction and Multiplication}
\index{pb\_add} \index{pb\_sub}
\begin{alltt}
int pb_add(pb_poly *a, pb_poly *b, pb_poly *c);
int pb_sub(pb_poly *a, pb_poly *b, pb_poly *c);
int pb_mul(pb_poly *a, pb_poly *b, pb_poly *c);
\end{alltt}

These will add (subtract or multiply, respectfully) the polynomial ``a'' and polynomial ``b'' and store the result in 
polynomial ``c''.  The characteristic from  ``c'' is used to calculate the result.  Note that the coefficients of ``c'' 
will always be positive provided the characteristic of ``c'' is greater than zero.  

Quick examples of usage.

\begin{alltt}
pb_add(a, b, c);  /* c = a + b */
pb_sub(b, a, c);  /* c = b - a */
pb_mul(c, a, a);  /* a = c * a */
\end{alltt}

\section{Division}
\index{pb\_div}
\begin{alltt}
int pb_div(pb_poly *a, pb_poly *b, pb_poly *c, pb_poly *d);
\end{alltt}
This will divide the polynomial ``a'' by ``b'' and store the quotient in ``c'' and remainder in ``d''.  That is

\begin{equation}
b(x) \cdot c(x) + d(x) = a(x)
\end{equation}

The value of $deg(d(x))$ is always less than $deg(b(x))$.  Either of ``c'' and ``d'' can be set to \textbf{NULL} to 
signify their value is not desired.  This is useful if you only want the quotient or remainder but not both.  

Since one of the destinations can be \textbf{NULL} the characteristic of the result is taken from ``b''.  The function
will return an error if the characteristic of ``a'' differs from that of ``b''.  

This function is defined for polynomials in $GF(p)[x]$ only.  A routine pb\_zdiv()\footnote{To be written!} allows the 
division of polynomials in $\Z[x]$.  

\section{Modular Functions}
\index{pb\_addmod} \index{pb\_submod} \index{pb\_mulmod}
\begin{alltt}
int pb_addmod(pb_poly *a, pb_poly *b, pb_poly *c, pb_poly *d);
int pb_submod(pb_poly *a, pb_poly *b, pb_poly *c, pb_poly *d);
int pb_mulmod(pb_poly *a, pb_poly *b, pb_poly *c, pb_poly *d);
\end{alltt}

These add (subtract or multiply respectfully) the polynomial ``a'' and the polynomial ``b'' modulo the polynomial ``c''
and store the result in the polynomial ``d''.  

\chapter{Algebraic Functions}

\section{Monic Reductions}
\index{pb\_monic}
\begin{alltt}
int pb_monic(pb_poly *a, pb_poly *b)
\end{alltt}
Makes ``b'' the monic representation of ``a'' by ensuring the most significant coefficient is one.  Only defined
over $GF(p)[x]$.  Note that this is not a straight copy to ``b'' so you must ensure the characteristic of the two
are equal before you call the function\footnote{Note that $a == b$ is acceptable as well.}.  Monic polynomials
are related to their original polynomial through an integer $k$ as follows

\begin{equation}
a(x) \cdot k^{-1} \equiv b(x)
\end{equation}

\section{Extended Euclidean Algorithm}
\index{pb\_exteuclid}
\begin{alltt}
int pb_exteuclid(pb_poly *a, pb_poly *b, 
                 pb_poly *U1, pb_poly *U2, pb_poly *U3);
\end{alltt}

This will compute the Euclidean algorithm and find values ``U1'', ``U2'', ``U3'' such that 

\begin{equation}
a(x) \cdot U1(x) + b(x) \cdot U2(x) = U3(x)
\end{equation}

The value of ``U3'' is reduced to a monic polynomial.  The three destination variables are all optional and can
be specified as \textbf{NULL} if they are not desired.

\section{Greatest Common Divisor}
\index{pb\_gcd}
\begin{alltt}
int pb_gcd(pb_poly *a, pb_poly *b, pb_poly *c);
\end{alltt}
This finds the monic greatest common divisor of the two polynomials ``a'' and ``b'' and store the result in ``c''.  The
operation is only defined over $GF(p)[x]$.  

\section{Modular Inverse}
\index{pb\_invmod}
\begin{alltt}
int pb_invmod(pb_poly *a, pb_poly *b, pb_poly *c);
\end{alltt}
This finds the modular inverse of ``a'' modulo ``b'' and stores the result in ``c''.  The operation is only defined over 
$GF(p)[x]$.  If the operation succeed then the following congruency should hold true.

\begin{equation}
a(x)c(x) \equiv 1 \mbox{ (mod }b(x)\mbox{)}
\end{equation}

\section{Modular Exponentiation}
\index{pb\_exptmod}
\begin{alltt}
int pb_exptmod (pb_poly * G, mp_int * X, pb_poly * P, pb_poly * Y);
\end{alltt}

This raise ``G'' to the power of ``X'' modulo ``P'' and stores the result in ``Y''. Or as a congruence

\begin{equation}
Y(x) \equiv G(x)^X \mbox{ (mod }P(x)\mbox{)}
\end{equation}

Where ``X'' can be negative\footnote{But in that case $G^{-1}(x)$ must exist modulo $P(x)$.} or positive.  This function
is only defined over $GF(p)[x]$.  

\section{Irreducibility Testing}
\index{pb\_isirreduc}
\begin{alltt}
int pb_isirreduc(pb_poly *a, int *res);
\end{alltt}
Sets ``res'' to MP\_YES if ``a'' is irreducible (only for $GF(p)[x]$) otherwise sets ``res'' to MP\_NO.  

\documentclass[b5paper]{book}
\usepackage{hyperref}
\usepackage{makeidx}
\usepackage{amssymb}
\usepackage{color}
\usepackage{alltt}
\usepackage{graphicx}
\usepackage{layout}
\def\union{\cup}
\def\intersect{\cap}
\def\getsrandom{\stackrel{\rm R}{\gets}}
\def\cross{\times}
\def\cat{\hspace{0.5em} \| \hspace{0.5em}}
\def\catn{$\|$}
\def\divides{\hspace{0.3em} | \hspace{0.3em}}
\def\nequiv{\not\equiv}
\def\approx{\raisebox{0.2ex}{\mbox{\small $\sim$}}}
\def\lcm{{\rm lcm}}
\def\gcd{{\rm gcd}}
\def\log{{\rm log}}
\def\ord{{\rm ord}}
\def\abs{{\mathit abs}}
\def\rep{{\mathit rep}}
\def\mod{{\mathit\ mod\ }}
\renewcommand{\pmod}[1]{\ ({\rm mod\ }{#1})}
\newcommand{\floor}[1]{\left\lfloor{#1}\right\rfloor}
\newcommand{\ceil}[1]{\left\lceil{#1}\right\rceil}
\def\Or{{\rm\ or\ }}
\def\And{{\rm\ and\ }}
\def\iff{\hspace{1em}\Longleftrightarrow\hspace{1em}}
\def\implies{\Rightarrow}
\def\undefined{{\rm ``undefined"}}
\def\Proof{\vspace{1ex}\noindent {\bf Proof:}\hspace{1em}}
\let\oldphi\phi
\def\phi{\varphi}
\def\Pr{{\rm Pr}}
\newcommand{\str}[1]{{\mathbf{#1}}}
\def\F{{\mathbb F}}
\def\N{{\mathbb N}}
\def\Z{{\mathbb Z}}
\def\R{{\mathbb R}}
\def\C{{\mathbb C}}
\def\Q{{\mathbb Q}}
\definecolor{DGray}{gray}{0.5}
\newcommand{\emailaddr}[1]{\mbox{$<${#1}$>$}}
\def\twiddle{\raisebox{0.3ex}{\mbox{\tiny $\sim$}}}
\def\gap{\vspace{0.5ex}}
\makeindex
\begin{document}
\frontmatter
\pagestyle{empty}
\title{LibTomPoly User Manual \\ v0.04}
\author{Tom St Denis \\ tomstdenis@iahu.ca}
\maketitle
This text and library are  hereby placed in the public domain.  This book has been 
formatted for B5 [176x250] paper using the \LaTeX{} {\em book} macro package.

\vspace{10cm}

\begin{flushright}Open Source.  Open Academia.  Open Minds.

\mbox{ }

Tom St Denis,

Ontario, Canada
\end{flushright}

\tableofcontents
\listoffigures
\mainmatter
\pagestyle{headings}
\chapter{Introduction}
\section{What is LibTomPoly?}
LibTomPoly is a public domain open source library to provide polynomial basis arithmetic.  It uses the public domain
library LibTomMath (not included) for the integer arithmetic and extends the functonality to provide polynomial arithmetic.

Technically speaking the library allows the user to perform arithmetic on elements from the group $GF(p)[x]$ and to 
a lesser extent (this will change in the future) over $\Z[x]$.  Essentially the math you can do with integers (including
forming rings and fields) you can do with with polynomials and now you can do with LibTomPoly.

\section{License}
LibTomPoly is public domain.  Enjoy.

\section{Terminology}
Throughout this manual and within the library there will be some terminology that not everyone is familiar with.  It is afterall
weird math.

\begin{figure}[here]
\begin{center}
\begin{small}
\begin{tabular}{|l|l|}
\hline \textbf{Term} & \textbf{Definition} \\
\hline monic polynomial & A polynomial where the leading coefficient is a one. \\
\hline irreducible polynomial & A polynomial that has no factors in a given group.  \\
                              & For instance, $x^2 + 4$ is irreducible in $\Z[x]$ but not \\
                              & in $GF(17)[x]$ since $x^2 + 4 = (x+8)(x+9) \mbox{ (mod }17\mbox{)}$.  \\
\hline primitive polynomial   & An irreducible polynomial which generates all of \\
                              & elements of a given field (e.g. $GF(p)[x]/v(x)$) \\
\hline characteristic         & An integer $k$ such that $k \cdot p(x) \equiv 0$  \\
\hline deg()                  & Functon returns degree of polynomial, e.g. deg($x^6 + x^3 + 1$) = 6 \\
\hline
\end{tabular}
\end{small}
\end{center}
\caption{Terminology}
\end{figure} 

\section{Building the Library}
The library is not ready for production yet but you can test out the library manually if you want.  To build the library 
simply type

\begin{alltt}
make
\end{alltt}

Which will build ``libtompoly.a''.  To build a Win32 library with MSVC type

\begin{alltt}
nmake -f makefile.msvc
\end{alltt}

To build against this library include ``tompoly.h'' and link against ``libtompoly.a'' (or tommath.lib as appropriate).
To build the included demo type 

\begin{alltt}
make demo
\end{alltt}

Which will build ``demo'' in the current directory.  The demo is not interactive and produces results which must be manually
inspected.

\chapter{Getting Started}

\section{The LibTomMath Connection}
LibTomPoly is really just an extension of LibTomMath\footnote{\url{http://math.libtomcrypt.org}}.  As such the library has 
been designed in much the same way as far as argument passing and error handling events are concerned.  The reader is 
encouraged to become familiar with LibTomMath before diving into LibTomPoly.

\section{The pb\_poly structure}
A polynomial can characterized by a few variables.  Given the C structure as follows

\begin{alltt}
typedef struct \{
   int    used,                  /* number of terms */
          alloc;                 /* number of terms available (total) */
   mp_int characteristic,        /* characteristic, zero if not finite */
          *terms;                /* terms of polynomial */
\} pb_poly;
\end{alltt}

\begin{enumerate}
   \item The \textbf{used} member indicates how many terms of the \textbf{terms} array are used to represent the polynomial.
   \item The \textbf{alloc} member indicates the size of the \textbf{terms} array.  Also note that even if \textbf{used}
         is less than \textbf{alloc} the mp\_ints above \textbf{used} in the array must be set to a valid representation 
         of zero.
   \item The \textbf{characteristic} member is an mp\_int representing the characteristic of the polynomial.  If the desire is to
         have a null characteristic (e.g. $\Z[x]$) this element must still be initialized to a valid representation
         of zero. 
   \item The \textbf{terms} member is a dynamically sized array of mp\_int values which represent the coefficients for
         the terms of the polynomial.  They start from least to most significant degree.  E.g. $p(x) = \sum_{i=0}^{used-1} terms_i \cdot x^i$.
\end{enumerate}

\section{Return Codes}
The library uses the return codes from LibTomMath.  They are

\index{MP\_OKAY}\index{MP\_YES}\index{MP\_NO}\index{MP\_VAL}\index{MP\_MEM}
\begin{figure}[here!]
\begin{center}
\begin{small}
\begin{tabular}{|l|l|}
\hline \textbf{Code} & \textbf{Meaning} \\
\hline MP\_OKAY & The function succeeded. \\
\hline MP\_VAL  & The function input was invalid. \\
\hline MP\_MEM  & Heap memory exhausted. \\
\hline &\\
\hline MP\_YES  & Response is yes. \\
\hline MP\_NO   & Response is no. \\
\hline
\end{tabular}
\end{small}
\end{center}
\caption{Return Codes}
\end{figure}

\section{Function Argument Passing}
Just like LibTomMath the arguments are meant to be read left to right where the destination is on the right.  Consider
the following.

\begin{alltt}
pb_add(a, b, c);   /* c = a + b */
pb_mul(a, b, c);   /* c = a * b */
\end{alltt}

Also like LibTomMath input arguments can be specified as output arguments.  Consider.

\begin{alltt}
pb_mul(a, b, a);   /* a = a * b */
pb_gcd(a, b, b);   /* b = (a, b) */
\end{alltt}

However, polynomial math raises another consideration.  The characteristic of the result is taken from the right most
argument passed to the function.  Not all functions will return an error code if the characteristics of the inputs
do not match so it's important to keep this in mind.  In general the results are undefined if not all of the polynomials
have identical characteristics.

\section{Initializing Polynomials}
In order to use a pb\_poly structure with one of the functions in this library the structure must be initialized.  
There are three functions provided to initialize pb\_poly structures.

\subsection{Default Initialization}
\index{pb\_init}
\begin{alltt}
int pb_init(pb_poly *a, mp_int *characteristic);
\end{alltt}
This will initialize ``a'' with the given ``characteristic'' such that the polynomial represented is a constant zero.  
The mp\_int characteristic must be a valid initialized mp\_int even if a characteristic of zero is desired.  By default, 
the polynomial will be initialized so there are ``PB\_TERMS'' terms available.  This will grow automatically as required 
by the other functions.

\subsection{Initilization of Given Size}
\index{pb\_init\_size}
\begin{alltt}
int pb_init_size(pb_poly *a, mp_int *characteristic, int size);
\end{alltt}
This behaves similar to pb\_init() except it will allocate ``size'' terms to initialize instead of ``PB\_TERMS''.  This
is useful if you happen to know in advance how many terms you want.

\subsection{Initilization of a Copy}
\index{pb\_init\_copy}
\begin{alltt}
int pb_init_copy(pb_poly *a, pb_poly *b);
\end{alltt}

This will initialize ``a'' so it is a verbatim copy of ``b''.  It will copy the characteristic and all of the terms
from ``b'' into ``a''.

\subsection{Freeing a Polynomial}
\index{pb\_clear}
\begin{alltt}
int pb_clear(pb_poly *a);
\end{alltt}

This will free all the memory required by ``a'' and mark it as been freed.  You can repeatedly pb\_clear() the same
pb\_poly safely.

\chapter{Basic Operations}
\section{Comparison}
Comparisions with polynomials is a bit less intuitive then with integers.  Is $x^2 + 3$ greater than $x^2 + x + 4$?  To
create a rational form of comparison the following comparison codes were designed.

\begin{figure}[here]
\begin{small}
\begin{center}
\begin{tabular}{|l|l|}
\hline \textbf{Code} & \textbf{Meaning} \\
\hline PB\_EQ & The polynomials are exactly equal \\
\hline PB\_DEG\_LT & The left polynomial has a lower degree than the right. \\
\hline PB\_DEG\_EQ & Both have the same degree. \\
\hline PB\_DEG\_GT & The left polynomial has a higher degree than the right. \\
\hline
\end{tabular}
\end{center}
\end{small}
\caption{Compare Codes}
\end{figure}

\index{pb\_cmp}
\begin{alltt}
int pb_cmp(pb_poly *a, pb_poly *b);
\end{alltt}

This will compare the polynomial ``a'' to the left of the polynomial ``b''.  It will return one of the four
codes listed above.  Note that the function does not compare the characteristics.  So if $a \in GF(17)[x]$ and 
$b \in GF(11)[x]$ were both equal to $x^2 + 3$ they would compare to PB\_EQ.  Whereas $x^3 + 4$
would compare to PB\_DEG\_LT, $x^1 + 7$ would compare to $PB\_DEG\_GT$ and $x^2 + 7$ would compare to $PB\_DEG\_EQ$\footnote{If the polynomial $a$ were on the left for all three cases.}.

\section{Copying and Swapping}
\index{pb\_copy}
\begin{alltt}
int pb_copy(pb_poly *src, pb_poly *dest);
\end{alltt}
This will copy the polynomial from ``src'' to ``dest'' verbatim.

\index{pb\_exch}
\begin{alltt}
int pb_exch(pb_poly *a, pb_poly *b);
\end{alltt}
This will exchange the contents of ``a'' with ``b''.  

\section{Multiplying and Dividing by $x$}
\index{pb\_lshd} \index{pb\_rshd}
\begin{alltt}
int pb_lshd(pb_poly *a, int i);
int pb_rshd(pb_poly *a, int i);
\end{alltt}
These will multiply (or divide, respectfully) the polynomial ``a'' by $x^i$.  If $i \le 0$ the functions return without
performing any operation.  For example,

\begin{alltt}
pb_lshd(a, 2);  /* a(x) = a(x) * x^2 */
pb_rshd(a, 7);  /* a(x) = a(x) / x^7 */
\end{alltt}

\chapter{Basic Arithmetic}
\section{Addition, Subtraction and Multiplication}
\index{pb\_add} \index{pb\_sub}
\begin{alltt}
int pb_add(pb_poly *a, pb_poly *b, pb_poly *c);
int pb_sub(pb_poly *a, pb_poly *b, pb_poly *c);
int pb_mul(pb_poly *a, pb_poly *b, pb_poly *c);
\end{alltt}

These will add (subtract or multiply, respectfully) the polynomial ``a'' and polynomial ``b'' and store the result in 
polynomial ``c''.  The characteristic from  ``c'' is used to calculate the result.  Note that the coefficients of ``c'' 
will always be positive provided the characteristic of ``c'' is greater than zero.  

Quick examples of usage.

\begin{alltt}
pb_add(a, b, c);  /* c = a + b */
pb_sub(b, a, c);  /* c = b - a */
pb_mul(c, a, a);  /* a = c * a */
\end{alltt}

\section{Division}
\index{pb\_div}
\begin{alltt}
int pb_div(pb_poly *a, pb_poly *b, pb_poly *c, pb_poly *d);
\end{alltt}
This will divide the polynomial ``a'' by ``b'' and store the quotient in ``c'' and remainder in ``d''.  That is

\begin{equation}
b(x) \cdot c(x) + d(x) = a(x)
\end{equation}

The value of $deg(d(x))$ is always less than $deg(b(x))$.  Either of ``c'' and ``d'' can be set to \textbf{NULL} to 
signify their value is not desired.  This is useful if you only want the quotient or remainder but not both.  

Since one of the destinations can be \textbf{NULL} the characteristic of the result is taken from ``b''.  The function
will return an error if the characteristic of ``a'' differs from that of ``b''.  

This function is defined for polynomials in $GF(p)[x]$ only.  A routine pb\_zdiv()\footnote{To be written!} allows the 
division of polynomials in $\Z[x]$.  

\section{Modular Functions}
\index{pb\_addmod} \index{pb\_submod} \index{pb\_mulmod}
\begin{alltt}
int pb_addmod(pb_poly *a, pb_poly *b, pb_poly *c, pb_poly *d);
int pb_submod(pb_poly *a, pb_poly *b, pb_poly *c, pb_poly *d);
int pb_mulmod(pb_poly *a, pb_poly *b, pb_poly *c, pb_poly *d);
\end{alltt}

These add (subtract or multiply respectfully) the polynomial ``a'' and the polynomial ``b'' modulo the polynomial ``c''
and store the result in the polynomial ``d''.  

\chapter{Algebraic Functions}

\section{Monic Reductions}
\index{pb\_monic}
\begin{alltt}
int pb_monic(pb_poly *a, pb_poly *b)
\end{alltt}
Makes ``b'' the monic representation of ``a'' by ensuring the most significant coefficient is one.  Only defined
over $GF(p)[x]$.  Note that this is not a straight copy to ``b'' so you must ensure the characteristic of the two
are equal before you call the function\footnote{Note that $a == b$ is acceptable as well.}.  Monic polynomials
are related to their original polynomial through an integer $k$ as follows

\begin{equation}
a(x) \cdot k^{-1} \equiv b(x)
\end{equation}

\section{Extended Euclidean Algorithm}
\index{pb\_exteuclid}
\begin{alltt}
int pb_exteuclid(pb_poly *a, pb_poly *b, 
                 pb_poly *U1, pb_poly *U2, pb_poly *U3);
\end{alltt}

This will compute the Euclidean algorithm and find values ``U1'', ``U2'', ``U3'' such that 

\begin{equation}
a(x) \cdot U1(x) + b(x) \cdot U2(x) = U3(x)
\end{equation}

The value of ``U3'' is reduced to a monic polynomial.  The three destination variables are all optional and can
be specified as \textbf{NULL} if they are not desired.

\section{Greatest Common Divisor}
\index{pb\_gcd}
\begin{alltt}
int pb_gcd(pb_poly *a, pb_poly *b, pb_poly *c);
\end{alltt}
This finds the monic greatest common divisor of the two polynomials ``a'' and ``b'' and store the result in ``c''.  The
operation is only defined over $GF(p)[x]$.  

\section{Modular Inverse}
\index{pb\_invmod}
\begin{alltt}
int pb_invmod(pb_poly *a, pb_poly *b, pb_poly *c);
\end{alltt}
This finds the modular inverse of ``a'' modulo ``b'' and stores the result in ``c''.  The operation is only defined over 
$GF(p)[x]$.  If the operation succeed then the following congruency should hold true.

\begin{equation}
a(x)c(x) \equiv 1 \mbox{ (mod }b(x)\mbox{)}
\end{equation}

\section{Modular Exponentiation}
\index{pb\_exptmod}
\begin{alltt}
int pb_exptmod (pb_poly * G, mp_int * X, pb_poly * P, pb_poly * Y);
\end{alltt}

This raise ``G'' to the power of ``X'' modulo ``P'' and stores the result in ``Y''. Or as a congruence

\begin{equation}
Y(x) \equiv G(x)^X \mbox{ (mod }P(x)\mbox{)}
\end{equation}

Where ``X'' can be negative\footnote{But in that case $G^{-1}(x)$ must exist modulo $P(x)$.} or positive.  This function
is only defined over $GF(p)[x]$.  

\section{Irreducibility Testing}
\index{pb\_isirreduc}
\begin{alltt}
int pb_isirreduc(pb_poly *a, int *res);
\end{alltt}
Sets ``res'' to MP\_YES if ``a'' is irreducible (only for $GF(p)[x]$) otherwise sets ``res'' to MP\_NO.  

\documentclass[b5paper]{book}
\usepackage{hyperref}
\usepackage{makeidx}
\usepackage{amssymb}
\usepackage{color}
\usepackage{alltt}
\usepackage{graphicx}
\usepackage{layout}
\def\union{\cup}
\def\intersect{\cap}
\def\getsrandom{\stackrel{\rm R}{\gets}}
\def\cross{\times}
\def\cat{\hspace{0.5em} \| \hspace{0.5em}}
\def\catn{$\|$}
\def\divides{\hspace{0.3em} | \hspace{0.3em}}
\def\nequiv{\not\equiv}
\def\approx{\raisebox{0.2ex}{\mbox{\small $\sim$}}}
\def\lcm{{\rm lcm}}
\def\gcd{{\rm gcd}}
\def\log{{\rm log}}
\def\ord{{\rm ord}}
\def\abs{{\mathit abs}}
\def\rep{{\mathit rep}}
\def\mod{{\mathit\ mod\ }}
\renewcommand{\pmod}[1]{\ ({\rm mod\ }{#1})}
\newcommand{\floor}[1]{\left\lfloor{#1}\right\rfloor}
\newcommand{\ceil}[1]{\left\lceil{#1}\right\rceil}
\def\Or{{\rm\ or\ }}
\def\And{{\rm\ and\ }}
\def\iff{\hspace{1em}\Longleftrightarrow\hspace{1em}}
\def\implies{\Rightarrow}
\def\undefined{{\rm ``undefined"}}
\def\Proof{\vspace{1ex}\noindent {\bf Proof:}\hspace{1em}}
\let\oldphi\phi
\def\phi{\varphi}
\def\Pr{{\rm Pr}}
\newcommand{\str}[1]{{\mathbf{#1}}}
\def\F{{\mathbb F}}
\def\N{{\mathbb N}}
\def\Z{{\mathbb Z}}
\def\R{{\mathbb R}}
\def\C{{\mathbb C}}
\def\Q{{\mathbb Q}}
\definecolor{DGray}{gray}{0.5}
\newcommand{\emailaddr}[1]{\mbox{$<${#1}$>$}}
\def\twiddle{\raisebox{0.3ex}{\mbox{\tiny $\sim$}}}
\def\gap{\vspace{0.5ex}}
\makeindex
\begin{document}
\frontmatter
\pagestyle{empty}
\title{LibTomPoly User Manual \\ v0.04}
\author{Tom St Denis \\ tomstdenis@iahu.ca}
\maketitle
This text and library are  hereby placed in the public domain.  This book has been 
formatted for B5 [176x250] paper using the \LaTeX{} {\em book} macro package.

\vspace{10cm}

\begin{flushright}Open Source.  Open Academia.  Open Minds.

\mbox{ }

Tom St Denis,

Ontario, Canada
\end{flushright}

\tableofcontents
\listoffigures
\mainmatter
\pagestyle{headings}
\chapter{Introduction}
\section{What is LibTomPoly?}
LibTomPoly is a public domain open source library to provide polynomial basis arithmetic.  It uses the public domain
library LibTomMath (not included) for the integer arithmetic and extends the functonality to provide polynomial arithmetic.

Technically speaking the library allows the user to perform arithmetic on elements from the group $GF(p)[x]$ and to 
a lesser extent (this will change in the future) over $\Z[x]$.  Essentially the math you can do with integers (including
forming rings and fields) you can do with with polynomials and now you can do with LibTomPoly.

\section{License}
LibTomPoly is public domain.  Enjoy.

\section{Terminology}
Throughout this manual and within the library there will be some terminology that not everyone is familiar with.  It is afterall
weird math.

\begin{figure}[here]
\begin{center}
\begin{small}
\begin{tabular}{|l|l|}
\hline \textbf{Term} & \textbf{Definition} \\
\hline monic polynomial & A polynomial where the leading coefficient is a one. \\
\hline irreducible polynomial & A polynomial that has no factors in a given group.  \\
                              & For instance, $x^2 + 4$ is irreducible in $\Z[x]$ but not \\
                              & in $GF(17)[x]$ since $x^2 + 4 = (x+8)(x+9) \mbox{ (mod }17\mbox{)}$.  \\
\hline primitive polynomial   & An irreducible polynomial which generates all of \\
                              & elements of a given field (e.g. $GF(p)[x]/v(x)$) \\
\hline characteristic         & An integer $k$ such that $k \cdot p(x) \equiv 0$  \\
\hline deg()                  & Functon returns degree of polynomial, e.g. deg($x^6 + x^3 + 1$) = 6 \\
\hline
\end{tabular}
\end{small}
\end{center}
\caption{Terminology}
\end{figure} 

\section{Building the Library}
The library is not ready for production yet but you can test out the library manually if you want.  To build the library 
simply type

\begin{alltt}
make
\end{alltt}

Which will build ``libtompoly.a''.  To build a Win32 library with MSVC type

\begin{alltt}
nmake -f makefile.msvc
\end{alltt}

To build against this library include ``tompoly.h'' and link against ``libtompoly.a'' (or tommath.lib as appropriate).
To build the included demo type 

\begin{alltt}
make demo
\end{alltt}

Which will build ``demo'' in the current directory.  The demo is not interactive and produces results which must be manually
inspected.

\chapter{Getting Started}

\section{The LibTomMath Connection}
LibTomPoly is really just an extension of LibTomMath\footnote{\url{http://math.libtomcrypt.org}}.  As such the library has 
been designed in much the same way as far as argument passing and error handling events are concerned.  The reader is 
encouraged to become familiar with LibTomMath before diving into LibTomPoly.

\section{The pb\_poly structure}
A polynomial can characterized by a few variables.  Given the C structure as follows

\begin{alltt}
typedef struct \{
   int    used,                  /* number of terms */
          alloc;                 /* number of terms available (total) */
   mp_int characteristic,        /* characteristic, zero if not finite */
          *terms;                /* terms of polynomial */
\} pb_poly;
\end{alltt}

\begin{enumerate}
   \item The \textbf{used} member indicates how many terms of the \textbf{terms} array are used to represent the polynomial.
   \item The \textbf{alloc} member indicates the size of the \textbf{terms} array.  Also note that even if \textbf{used}
         is less than \textbf{alloc} the mp\_ints above \textbf{used} in the array must be set to a valid representation 
         of zero.
   \item The \textbf{characteristic} member is an mp\_int representing the characteristic of the polynomial.  If the desire is to
         have a null characteristic (e.g. $\Z[x]$) this element must still be initialized to a valid representation
         of zero. 
   \item The \textbf{terms} member is a dynamically sized array of mp\_int values which represent the coefficients for
         the terms of the polynomial.  They start from least to most significant degree.  E.g. $p(x) = \sum_{i=0}^{used-1} terms_i \cdot x^i$.
\end{enumerate}

\section{Return Codes}
The library uses the return codes from LibTomMath.  They are

\index{MP\_OKAY}\index{MP\_YES}\index{MP\_NO}\index{MP\_VAL}\index{MP\_MEM}
\begin{figure}[here!]
\begin{center}
\begin{small}
\begin{tabular}{|l|l|}
\hline \textbf{Code} & \textbf{Meaning} \\
\hline MP\_OKAY & The function succeeded. \\
\hline MP\_VAL  & The function input was invalid. \\
\hline MP\_MEM  & Heap memory exhausted. \\
\hline &\\
\hline MP\_YES  & Response is yes. \\
\hline MP\_NO   & Response is no. \\
\hline
\end{tabular}
\end{small}
\end{center}
\caption{Return Codes}
\end{figure}

\section{Function Argument Passing}
Just like LibTomMath the arguments are meant to be read left to right where the destination is on the right.  Consider
the following.

\begin{alltt}
pb_add(a, b, c);   /* c = a + b */
pb_mul(a, b, c);   /* c = a * b */
\end{alltt}

Also like LibTomMath input arguments can be specified as output arguments.  Consider.

\begin{alltt}
pb_mul(a, b, a);   /* a = a * b */
pb_gcd(a, b, b);   /* b = (a, b) */
\end{alltt}

However, polynomial math raises another consideration.  The characteristic of the result is taken from the right most
argument passed to the function.  Not all functions will return an error code if the characteristics of the inputs
do not match so it's important to keep this in mind.  In general the results are undefined if not all of the polynomials
have identical characteristics.

\section{Initializing Polynomials}
In order to use a pb\_poly structure with one of the functions in this library the structure must be initialized.  
There are three functions provided to initialize pb\_poly structures.

\subsection{Default Initialization}
\index{pb\_init}
\begin{alltt}
int pb_init(pb_poly *a, mp_int *characteristic);
\end{alltt}
This will initialize ``a'' with the given ``characteristic'' such that the polynomial represented is a constant zero.  
The mp\_int characteristic must be a valid initialized mp\_int even if a characteristic of zero is desired.  By default, 
the polynomial will be initialized so there are ``PB\_TERMS'' terms available.  This will grow automatically as required 
by the other functions.

\subsection{Initilization of Given Size}
\index{pb\_init\_size}
\begin{alltt}
int pb_init_size(pb_poly *a, mp_int *characteristic, int size);
\end{alltt}
This behaves similar to pb\_init() except it will allocate ``size'' terms to initialize instead of ``PB\_TERMS''.  This
is useful if you happen to know in advance how many terms you want.

\subsection{Initilization of a Copy}
\index{pb\_init\_copy}
\begin{alltt}
int pb_init_copy(pb_poly *a, pb_poly *b);
\end{alltt}

This will initialize ``a'' so it is a verbatim copy of ``b''.  It will copy the characteristic and all of the terms
from ``b'' into ``a''.

\subsection{Freeing a Polynomial}
\index{pb\_clear}
\begin{alltt}
int pb_clear(pb_poly *a);
\end{alltt}

This will free all the memory required by ``a'' and mark it as been freed.  You can repeatedly pb\_clear() the same
pb\_poly safely.

\chapter{Basic Operations}
\section{Comparison}
Comparisions with polynomials is a bit less intuitive then with integers.  Is $x^2 + 3$ greater than $x^2 + x + 4$?  To
create a rational form of comparison the following comparison codes were designed.

\begin{figure}[here]
\begin{small}
\begin{center}
\begin{tabular}{|l|l|}
\hline \textbf{Code} & \textbf{Meaning} \\
\hline PB\_EQ & The polynomials are exactly equal \\
\hline PB\_DEG\_LT & The left polynomial has a lower degree than the right. \\
\hline PB\_DEG\_EQ & Both have the same degree. \\
\hline PB\_DEG\_GT & The left polynomial has a higher degree than the right. \\
\hline
\end{tabular}
\end{center}
\end{small}
\caption{Compare Codes}
\end{figure}

\index{pb\_cmp}
\begin{alltt}
int pb_cmp(pb_poly *a, pb_poly *b);
\end{alltt}

This will compare the polynomial ``a'' to the left of the polynomial ``b''.  It will return one of the four
codes listed above.  Note that the function does not compare the characteristics.  So if $a \in GF(17)[x]$ and 
$b \in GF(11)[x]$ were both equal to $x^2 + 3$ they would compare to PB\_EQ.  Whereas $x^3 + 4$
would compare to PB\_DEG\_LT, $x^1 + 7$ would compare to $PB\_DEG\_GT$ and $x^2 + 7$ would compare to $PB\_DEG\_EQ$\footnote{If the polynomial $a$ were on the left for all three cases.}.

\section{Copying and Swapping}
\index{pb\_copy}
\begin{alltt}
int pb_copy(pb_poly *src, pb_poly *dest);
\end{alltt}
This will copy the polynomial from ``src'' to ``dest'' verbatim.

\index{pb\_exch}
\begin{alltt}
int pb_exch(pb_poly *a, pb_poly *b);
\end{alltt}
This will exchange the contents of ``a'' with ``b''.  

\section{Multiplying and Dividing by $x$}
\index{pb\_lshd} \index{pb\_rshd}
\begin{alltt}
int pb_lshd(pb_poly *a, int i);
int pb_rshd(pb_poly *a, int i);
\end{alltt}
These will multiply (or divide, respectfully) the polynomial ``a'' by $x^i$.  If $i \le 0$ the functions return without
performing any operation.  For example,

\begin{alltt}
pb_lshd(a, 2);  /* a(x) = a(x) * x^2 */
pb_rshd(a, 7);  /* a(x) = a(x) / x^7 */
\end{alltt}

\chapter{Basic Arithmetic}
\section{Addition, Subtraction and Multiplication}
\index{pb\_add} \index{pb\_sub}
\begin{alltt}
int pb_add(pb_poly *a, pb_poly *b, pb_poly *c);
int pb_sub(pb_poly *a, pb_poly *b, pb_poly *c);
int pb_mul(pb_poly *a, pb_poly *b, pb_poly *c);
\end{alltt}

These will add (subtract or multiply, respectfully) the polynomial ``a'' and polynomial ``b'' and store the result in 
polynomial ``c''.  The characteristic from  ``c'' is used to calculate the result.  Note that the coefficients of ``c'' 
will always be positive provided the characteristic of ``c'' is greater than zero.  

Quick examples of usage.

\begin{alltt}
pb_add(a, b, c);  /* c = a + b */
pb_sub(b, a, c);  /* c = b - a */
pb_mul(c, a, a);  /* a = c * a */
\end{alltt}

\section{Division}
\index{pb\_div}
\begin{alltt}
int pb_div(pb_poly *a, pb_poly *b, pb_poly *c, pb_poly *d);
\end{alltt}
This will divide the polynomial ``a'' by ``b'' and store the quotient in ``c'' and remainder in ``d''.  That is

\begin{equation}
b(x) \cdot c(x) + d(x) = a(x)
\end{equation}

The value of $deg(d(x))$ is always less than $deg(b(x))$.  Either of ``c'' and ``d'' can be set to \textbf{NULL} to 
signify their value is not desired.  This is useful if you only want the quotient or remainder but not both.  

Since one of the destinations can be \textbf{NULL} the characteristic of the result is taken from ``b''.  The function
will return an error if the characteristic of ``a'' differs from that of ``b''.  

This function is defined for polynomials in $GF(p)[x]$ only.  A routine pb\_zdiv()\footnote{To be written!} allows the 
division of polynomials in $\Z[x]$.  

\section{Modular Functions}
\index{pb\_addmod} \index{pb\_submod} \index{pb\_mulmod}
\begin{alltt}
int pb_addmod(pb_poly *a, pb_poly *b, pb_poly *c, pb_poly *d);
int pb_submod(pb_poly *a, pb_poly *b, pb_poly *c, pb_poly *d);
int pb_mulmod(pb_poly *a, pb_poly *b, pb_poly *c, pb_poly *d);
\end{alltt}

These add (subtract or multiply respectfully) the polynomial ``a'' and the polynomial ``b'' modulo the polynomial ``c''
and store the result in the polynomial ``d''.  

\chapter{Algebraic Functions}

\section{Monic Reductions}
\index{pb\_monic}
\begin{alltt}
int pb_monic(pb_poly *a, pb_poly *b)
\end{alltt}
Makes ``b'' the monic representation of ``a'' by ensuring the most significant coefficient is one.  Only defined
over $GF(p)[x]$.  Note that this is not a straight copy to ``b'' so you must ensure the characteristic of the two
are equal before you call the function\footnote{Note that $a == b$ is acceptable as well.}.  Monic polynomials
are related to their original polynomial through an integer $k$ as follows

\begin{equation}
a(x) \cdot k^{-1} \equiv b(x)
\end{equation}

\section{Extended Euclidean Algorithm}
\index{pb\_exteuclid}
\begin{alltt}
int pb_exteuclid(pb_poly *a, pb_poly *b, 
                 pb_poly *U1, pb_poly *U2, pb_poly *U3);
\end{alltt}

This will compute the Euclidean algorithm and find values ``U1'', ``U2'', ``U3'' such that 

\begin{equation}
a(x) \cdot U1(x) + b(x) \cdot U2(x) = U3(x)
\end{equation}

The value of ``U3'' is reduced to a monic polynomial.  The three destination variables are all optional and can
be specified as \textbf{NULL} if they are not desired.

\section{Greatest Common Divisor}
\index{pb\_gcd}
\begin{alltt}
int pb_gcd(pb_poly *a, pb_poly *b, pb_poly *c);
\end{alltt}
This finds the monic greatest common divisor of the two polynomials ``a'' and ``b'' and store the result in ``c''.  The
operation is only defined over $GF(p)[x]$.  

\section{Modular Inverse}
\index{pb\_invmod}
\begin{alltt}
int pb_invmod(pb_poly *a, pb_poly *b, pb_poly *c);
\end{alltt}
This finds the modular inverse of ``a'' modulo ``b'' and stores the result in ``c''.  The operation is only defined over 
$GF(p)[x]$.  If the operation succeed then the following congruency should hold true.

\begin{equation}
a(x)c(x) \equiv 1 \mbox{ (mod }b(x)\mbox{)}
\end{equation}

\section{Modular Exponentiation}
\index{pb\_exptmod}
\begin{alltt}
int pb_exptmod (pb_poly * G, mp_int * X, pb_poly * P, pb_poly * Y);
\end{alltt}

This raise ``G'' to the power of ``X'' modulo ``P'' and stores the result in ``Y''. Or as a congruence

\begin{equation}
Y(x) \equiv G(x)^X \mbox{ (mod }P(x)\mbox{)}
\end{equation}

Where ``X'' can be negative\footnote{But in that case $G^{-1}(x)$ must exist modulo $P(x)$.} or positive.  This function
is only defined over $GF(p)[x]$.  

\section{Irreducibility Testing}
\index{pb\_isirreduc}
\begin{alltt}
int pb_isirreduc(pb_poly *a, int *res);
\end{alltt}
Sets ``res'' to MP\_YES if ``a'' is irreducible (only for $GF(p)[x]$) otherwise sets ``res'' to MP\_NO.  

\input{pb.ind}

\end{document}


\end{document}


\end{document}


\end{document}
